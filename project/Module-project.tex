\documentclass{beamer}

%\usetheme{Madrid}
%\usetheme{Boadilla}
%\usetheme{default}
%\usetheme{Warsaw}
%\usetheme{Bergen}
%\usetheme{Frankfurt}
\usetheme{Darmstadt}

\setbeamercolor{normal text}{fg=white}
\setbeamertemplate{background canvas}[vertical shading] [top=black!95,bottom=black!65]

\definecolor{mypurple}{RGB}{207,78,64}
\usecolortheme[named=mypurple]{structure}

\definecolor{myorange}{RGB}{255,235,190}
\beamerboxesdeclarecolorscheme{orange}{orange}{myorange}

\definecolor{commandcolor}{RGB}{111,195,165}

\setbeamertemplate{footline}[page number]
%\setbeamercovered{transparent}
\setbeamercovered{invisible}
\setbeamertemplate{navigation symbols}{}

%\usepackage{musixtex}
\usepackage{multimedia}
\usepackage{graphicx}
\usepackage[utf8]{inputenc}
%\usepackage[T1]{fontenc}
\usepackage[french]{babel} 
%\usepackage[all]{xy}
%\usepackage{multirow}
%\usepackage{lmodern}
\usepackage{subfigure}
%\usepackage{ulem}
\usepackage{url}
\usepackage{hyperref}
\usepackage{verbatim}
\usepackage{xspace}
\usepackage{color}
\usepackage{xcolor}
\usepackage{rotating}
\usepackage{multicol}
\usepackage[export]{adjustbox}
\usepackage{textpos}
\usepackage{listings}
\usepackage{fontawesome}


\definecolor{mypurple}{RGB}{207,78,64}
\usecolortheme[named=mypurple]{structure}

\definecolor{myorange}{RGB}{255,235,190}
\beamerboxesdeclarecolorscheme{orange}{orange}{myorange}

\definecolor{dgreen}{RGB}{0,125,0}

\usepackage{tikz}
\usetikzlibrary{trees}

\setbeamertemplate{caption}[numbered] 

\newcommand{\setframetitle}[1]{\begin{center}
    \huge \textbf{#1}
\end{center}}


%% --------------

\title{Projet}
\subtitle{Atelier d'aide à la programmation}
\author{L\'eo \textsc{Baudouin}}
\institute{
  {\url{baudouin.leo @ gmail.com}}
}
\date{03-04 juin 2019}

%% --------------

\begin{document}

\begin{frame}
  \titlepage
\end{frame}

%-------

\section{}
\subsection{}

\begin{frame}[fragile]{Télécharger les sources}
  \begin{block}{Cloner le dép\^ot suivant}
    \url{https://github.com/lbaudouin/module-project.git}
  \end{block}
    \begin{block}{Supprimer le dossier git}
\textcolor{cyan}{\verb?rm -rf .git?} 
  \end{block}
\end{frame}

\begin{frame}{Exercice}  
    \begin{exampleblock}{Consignes}
	Par groupe de deux étudiants:
    \begin{enumerate}
    \item Créer un dépôt et le partager entre vous
    \item Lier le dossier local au dépôt
    \item Répartir les tâches
    \item Générer une bibliothèque statique
    \item Générer un exécutable et le corriger
    \item Générer et passer les tests unitaire
    \item Générer la documentation
    \item Générer l'intégration continue
%   \item Déployer la documentation sur gitlab
    \end{enumerate}
  \end{exampleblock}
\end{frame}

%-------------------------------------------------------------------
\end{document} 

