\documentclass{beamer}

\input{../style.tex}

\usepackage{ulem}

%% --------------

\title{Projet}
\subtitle{Atelier d'aide à la programmation}
\author{L\'eo \textsc{Baudouin}}
\institute{
  {\url{baudouin.leo @ gmail.com}}
}
\date{19-20 juin 2025}

%% --------------

\begin{document}

\begin{frame}
  \titlepage
\end{frame}

%-------

\section{}
\subsection{Projet C++}

\begin{frame}[fragile]{Télécharger les sources}
  \begin{block}{Forker le dép\^ot suivant sur gitlab}
    \url{https://gitlab.com/lbaudouin/module-project}
  \end{block}
  
\includegraphics[width=\linewidth]{images/fork.png}  
  
\end{frame}

\begin{frame}[fragile]{Exercice - C++}  
    \begin{exampleblock}{Consignes}
	    Par groupe de deux étudiants ou plus :
    \begin{enumerate}
    %\item Répartir les tâches dans l'onglet \verb?Projects?
    \item Générer une bibliothèque statique avec CMake
    \item Générer un exécutable avec CMake
    \item Signaler le bug dans l'onglet \verb?Issues?
    \item Le corriger et fermer le ticket
    \item Générer et passer les tests unitaire avec CMake
    \item \sout{Générer la documentation avec Doxygen}
    \item Générer l'intégration continue
    \item Ajouter l'executable dans un container Docker
%   \item Déployer la documentation sur gitlab
    \end{enumerate}
  \end{exampleblock}
\end{frame}


\subsection{Projet Python}

\begin{frame}[fragile]{Exercice - Python}
\begin{exampleblock}{Consignes}
  \begin{enumerate}
    \item Écrire un module Python permet de charger un fichier CSV et de le convertir en une image, ainsi que quelques statistiques
    \item Créer un script Python qui utilise ce module afin de générer un article scientifique en Latex à partir d'un CSV
    \item Créer un pyproject.toml et utiliser pip-compile pour générer le requirements.txt
    \item Écrire des tests unitaires pour ce module avec PyTest
    \item Corriger les éventuels bugs détectés par les tests
    \item Proposer des améliorations et les ajouter dans le Gitlab pour les attribuer à votre binôme
    \item Écrire un \texttt{Dockerfile} pour exécuter le script avec Docker
    \item \sout{Ajouter de la documentation Doxygen}
  \end{enumerate}
\end{exampleblock}
\end{frame}  
    


%-------------------------------------------------------------------
\end{document} 

