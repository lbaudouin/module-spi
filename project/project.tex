\documentclass{beamer}

\input{../style.tex}

\usepackage{ulem}

%% --------------

\title{Projet}
\subtitle{Atelier d'aide à la programmation}
\author{L\'eo \textsc{Baudouin}}
\institute{
  {\url{baudouin.leo @ gmail.com}}
}
\date{19-20 juin 2025}

%% --------------

\begin{document}

\begin{frame}
  \titlepage

\end{frame}

%-------

\section{}
\subsection{}

\begin{frame}[fragile]{Télécharger les sources}
  \begin{block}{Forker le dép\^ot suivant sur gitlab}
    \url{https://gitlab.com/lbaudouin/module-project}
  \end{block}
  
\includegraphics[width=\linewidth]{images/fork.png}  
  
\end{frame}

\begin{frame}[fragile]{Exercice}  
    \begin{exampleblock}{Consignes}
	Par groupe de deux étudiants:
    \begin{enumerate}
    %\item Répartir les tâches dans l'onglet \verb?Projects?
    \item Générer une bibliothèque statique avec CMake
    \item Générer un exécutable avec CMake
    \item Signaler le bug dans l'onglet \verb?Issues?
    \item Le corriger et fermer le ticket
    \item Générer et passer les tests unitaire avec CMake
    \item Générer la documentation avec Doxygen
    \item \sout{Générer l'intégration continue}
%   \item Déployer la documentation sur gitlab
    \end{enumerate}
  \end{exampleblock}
\end{frame}

%-------------------------------------------------------------------
\end{document} 

